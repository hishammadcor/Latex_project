\documentclass{book}
\usepackage{tabularray}
\UseTblrLibrary{siunitx}
\usepackage{xcolor}
\usepackage{fontspec}
\usepackage{geometry}
\usepackage{siunitx}
\usepackage{graphicx}

\definecolor{green40}{RGB}{196, 229, 218}

\geometry{
    a4paper,																									% set paper dimensions
    twoside,
%
    inner			= 2.50cm,																					% inner (twosided) = left	= lmargin
    outer			= 2.50cm,																					% outer (twosided) = right	= rmargin
    top				= 2.5cm,																					% top margin from top paper edge to upper textarea edge
    bottom			= 2.00cm,																					% bottom margin excluding footer
%
%	includehead,																								% do not include header into textarea
    headsep			= 1.00cm,																					% distance between textarea and header
    headheight		= 12pt,																						% needs to support 8pt and 10pt text
%
    nomarginpar,																								% no additional outer margins
}


% Define a new column type combining X and S
%\NewColumnType{Y}{X[co=1] S[table-format=3, group-separator={.}, group-four-digits = true, round-mode=places,
%    round-precision = 0,round-direction = down]}


%\NewColumnType{S}[1][]{Q[si={table-format=3.0, table-number-alignment=center, round-mode=places, round-precision=0, round-direction=down}, c, blue7]}

% Define a custom column type S that includes the specified siunitx options
\sisetup{
    output-decimal-marker={,},
    group-separator={.},
    group-four-digits = true,
    table-number-alignment = center,
    table-format=4.0,
    round-mode=places,
    round-precision = 0,
    round-direction = down,
    detect-all
}

% Define a new column type for tabularray
\NewColumnType{S}{>{\collectcell\si}X[co=1]<{\endcollectcell}}

\NewColumnType{A}{>{\raggedright}p{12.0cm}}

\NewColumnType{B}{>{\raggedright}X}
\begin{document}
    
%\sisetup{
%    output-decimal-marker={,}, % Sets the decimal marker to comma for German number formatting
%    group-separator={.},       % Groups numbers with a dot
%    group-four-digits = true,   % Groups numbers in fours for easier reading (common in Germany)
%    table-number-alignment = center,
%    table-format=4.0,
%    round-mode=places,
%    round-precision = 0,
%    round-direction = down,
%    detect-all
%}


\setmainfont{Segoe UI}
\fontsize{8pt}{10pt}
\selectfont
%
%\newcolumntype{Y}{>{\centering\arraybackslash}c} % Flexible first column with text aligned to the left
%\newcolumntype{F}{>{\raggedright\arraybackslash}l} % style 33 is not 4 cm its 10
%\newcolumntype{S}{>{\centering\arraybackslash\columncolor{green40}}c}
%\newcolumntype{G}{>{\begin{tabularx}{\linewidth}{S}}X<{\end{tabularx}}}

%\newcolumntype{E}{>{\centering\arraybackslash\columncolor{green40}}S[table-column-width=2.24cm]} % Changed from empty to 'c' to apply the color to a central aligned column
%\newcolumntype{Y}{>{\centering\arraybackslash}X} % Flexible first column with text aligned to the left
%\newcolumntype{F}{>{\raggedright\arraybackslash}p{4.0cm}} % style 33 is not 4 cm its 10
%\newcolumntype{S}{>{\centering\arraybackslash\columncolor{green40}}X} % Changed from empty to 'c' to apply the color to a central aligned column
%\newcolumntype{H}{>{\hsize=\dimexpr3\hsize+4\tabcolsep+2\arrayrulewidth\relax\centering\arraybackslash\columncolor{green40}}X} % Changed from empty to 'c' to apply the color to a central aligned column

%\newcolumntype{Y}{>{\centering\arraybackslash}X} % Flexible first column with text aligned to the left
%\newcolumntype{B}{>{\raggedright}p{12.0cm}} % Fixed-width second column
%\newcolumntype{A}{>{\raggedright\arraybackslash}p{10.0cm}} % style 33 is not 4 cm its 10
    \resizebox{\textwidth}{\begin{tblr}{
        width = \textwidth,
        colspec = {A S S S S},
        column{even} = {green40}}
        \textit{- befragte fortgeschrittene Masterstudierende (Lehramt)-} & \textit{M} & \textit{SD} & \textit{MD} & \textit{N} \\
        Ich besitze ein fundiertes Wissen zur fachdidaktischen Forschung. & 3,7 & 11,4 & 4 & 38 \\ \hline
        Ich kann fachwissenschaftliche Inhalte unter didaktischen Aspekten analysieren. & 4,2 & 1,3 & 4,5 & 38 \\ \hline
        Ich verfüge über ein fachdidaktisches Methodenrepertoire, um das Fach angemessen zu unterrichten. & 3,9 & 1,3 & 4 & 38 \\ \hline
        Die fachdidaktischen Veranstaltungen meines Faches 1 konkretisieren die Inhalte regelmäßig an Beispielen aus der schulischen Praxis. & 3,0 & 1,2 & 3 & 37 \\ \hline
        Die fachdidaktischen Veranstaltungen meines Faches 1 bieten mir die Möglichkeit, bildungswissenschaftliche Erkenntnisse praktisch anzuwenden. & 2,6 & 1,4 & 2 & 38 \\ \hline
        Die fachdidaktischen Veranstaltungen meines Faches 1 erlauben es mir, Erfahrungen aus meinen Praktika zu thematisieren. & 3,0 & 1,4 & 3 & 37 \\ \hline
        Die fachdidaktischen Veranstaltungen meines Faches 1 bieten Anknüpfungspunkte zu fachdidaktischen Inhalten. & 3,2 & 1,5 & 3 & 38 \\
    \end{tblr}}

    \vspace{1.0cm}

    \resizebox{\textwidth}{!}{
        \begin{tblr}{
        width = \linewidth,
        colspec = {B S S S S S},
        column{even} = {green40}}
        & \textit{2018} & \textit{2019} & \textit{2020} & \textit{2021} & \textit{2022} \\ \hline
        Personal & 4806,84 & 2506,91  & 3272,48  & 4521,43  & 5328,85 \\ \hline
        Sachkosten & 3117,47 & 5730,86  & 705,57  & 5076,89  & 4019,89 \\ \hline
        Investitionen & 1548,19 & 2505,91  & 1244,48  & 0 & 0 \\ \hline
        Gesamtsumme & 9472,5 & 10743,68  & 5222,53  & 9598,32  & 9348,74 \\
    \end{tblr}}

\end{document}